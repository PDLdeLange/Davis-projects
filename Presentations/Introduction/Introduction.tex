\documentclass{beamer}
\usepackage[latin1]{inputenc}
\usetheme{default}
\title{Introduction}
\author{Peter de Lange}
\institute{UC DAVIS}
\date{January 27, 2011}
\begin{document}

\begin{frame}
\titlepage
\end{frame}

\begin{frame}{About me}
		\begin{itemize}
		\item Graduate student from the TU Delft (Netherlands)
		\item Doing internship at the UC Davis bicycle laboratory
		\end{itemize}
\end{frame}

\begin{frame}{Graduation project}
Identifying rider controlling action in bicycling:
		\begin{itemize}
		\item We know a lot about the uncontrolled bicycle dynamics.
		\item However, in every day life, bicycle are most of the time controlled by human operators.
		\item Graduation project will be about identifying human rider control during bicycling
		\item To be more specific, we are especially interested in bike balancing.
	 \end{itemize}
\end{frame}

\begin{frame}{Project Description}
		\begin{itemize}
		\item Assisting Jason and Luke with performing measurements.
		\item Prepare system identification procedures for validating bicycle dynamics (Luke) and estimating rider control (Jason)
		\item Unfortunately I just arrived, so I can't tell you much about it yet.
		\end{itemize}
\end{frame}

\begin{frame}{Bicycle simulation}
Instead I will be talking about Matlab bicycle game I recently created. The following contents will be treated. 
		\begin{itemize}
		\item Methods
		\item User input
		\item Demonstration
		\item Discussion
		\item Future work
		\end{itemize}
\end{frame}

\begin{frame}{Methods}
		\begin{itemize}
		\item Simulink model (ODE-solving)
		\item Matlab Real-time windows target
				\begin{itemize}
				\item Enable real-time simulation by connecting Matlab directly to windows timer.
				\item Compilation of simulink code to C-code for faster runs.
				\item Analog joystick input supported
				\item Compatible with Simulink 3D animation toolbox.
				\end{itemize}
		\item Simulink 3D animation toolbox
				\begin{itemize}
						\item Enables connecting Matlab with a 3D VRML enviroment.
						\item VRML environments are easy to created using 3D software (e.g. 3D studio max).
				\end{itemize}
		\item Matlab GUI with visual indicators
		\end{itemize}
\end{frame}

\begin{frame}{User input}
		\begin{itemize}
		\item Linear benchmark bicycle equations programmed;
		\begin{align} 
		\mathbf{M}\ddot{\mathbf{q}} + v\mathbf{C}_1\dot{\mathbf{q} } + \left[g\mathbf{K}_0 + v^2\mathbf{K}_2\right]\mathbf{q} = \mathbf{f} \ ,
		\end{align}
		where $\mathbf{q} = \left[\phi , \ \delta \right]^T$ and $\mathbf{f} = \left[ T_\phi , \ T_\delta \right]^T$.
		\item User input:
				\begin{itemize}
				\item Velocity; $v$ [m/s]
				\item Steering torque; $T_{\delta}$ [Nm]
				\end{itemize}
		\item Lean action omitted, but would be interesting to include in the simulation.
		\end{itemize}
\end{frame}

\begin{frame}{Demonstration}
			Matlab bicycle simulation 
\end{frame}

\begin{frame}{Some discussion}
			\begin{itemize}
			\item Dynamic behavior of the bicycle changes as function of the forward velocity; $v$.
			\item Capsize instability easy to control.
			\item At low velocity the weave mode becomes instable and is very hard to control.
			\item Linear equations only valid for small angles.
			\item Adding visual cue about roll rate, makes control possible.
			\item Changing bike parameters would require changing both the Matlab and 3D model, which is a lot of work.
			\end{itemize}
\end{frame}

\begin{frame}{Future work}
			\begin{itemize}
			\item Include the nonlinear equations
			\item Force feedback interface to include proprioceptive feedback loops.
			\item Include leaning action.
			\item Experiment with different viewpoints (e.g. camera attached to bicycle).
			\item Automatic 3D model generation based on bike parameters.
			\item Multi-player bike balancing mayhem.
			\end{itemize}
\end{frame}


\end{document}