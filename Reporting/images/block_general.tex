\begin{figure}[ht]
\centering %

% We need layers to draw the block diagram
\pgfdeclarelayer{background}
\pgfdeclarelayer{foreground}
\pgfsetlayers{background,main,foreground}


\tikzstyle{block} = [draw,rectangle,thick,minimum height=2em,minimum width=2em]
\tikzstyle{bigblock} = [draw,rectangle,thick,minimum height=2em,minimum width=3em]
\tikzstyle{sum} = [draw,circle,inner sep=0mm,minimum size=2mm]
\tikzstyle{connector} = [->,thick]
\tikzstyle{line} = [thick]	

\tikzstyle{branch} = [circle,inner sep=0pt,minimum size=1mm,fill=black,draw=black]
\tikzstyle{guide} = []

\def \myneq {\skew{-2}\not =} % \neq alone skews the dash

\begin{tikzpicture}[scale=1, auto, >=stealth']

    %  Loop function
    \node[coordinate]												(input) 		{};
		\node[coordinate,right of=input]		(sumphi)	{};
		\node[block,right of =sumphi] 			(K) 					{${K}$};
		\node[coordinate,right of=K]				(Tdelta)		{$T_\delta$};
		\node[block,right of =Tdelta]				(H) 					{$\mathbf{H}$};
		\node[coordinate,right of =H]				(phi) 				{};
    \node[coordinate, right of=phi]			(output) 	{};
		
		% Feedback paths
		\node[coordinate,below of=H]			(fbH)				{};
		\node[coordinate,below of=K]			(fbK)				{};
		
		% Draw connectors
		\draw[connector] (input) -- node {$\phi_c$} (K);
		\draw[connector] (K) -- node {$T_\delta$} (H);
		\draw[connector] (H) -- node {$\phi$} (output);
		
		\draw[line] (H) -- (fbH)  -- node {$\mathbf{q}$}  (fbK);
		\draw[connector] (fbK) -- (K);
		
		
			
		%% Feedback nodes
		%\path (H.center)+(-0.75em,-0.65em) node (Hdelta) {}; 	\node[coordinate,below of=Hdelta](fbHdelta){};
		%\path (H.center)+(-0.25em,-0.65em) node (Hphid) {};		\node[coordinate,below of=Hphid](fbHphid) {};
		%\path (H.center)+( 0.25em,-0.65em) node (Hphi) {};			\node[coordinate,below of=Hphi](fbHphi){};
		%\path (H.center)+( 0.75em,-0.65em) node (Hpsi) {};			\node[coordinate,below of=Hpsi](fbHpsi){};
		%\path (y.center)+( 0.00em,-0.65em) node (Hy) {};	  	\node[coordinate,below of=Hy](fbHy){};
%
		%% Multiple loop feedback
		%\draw [line] (Hdelta) -- 	node {} ([yshift=-0em] fbHdelta) ;
		%\draw [line] (Hphid) -- 		node {} ([yshift=-1em] fbHphid) ;
		%\draw [line] (Hphi) -- 			node {} ([yshift=-2em] fbHphi) ;
		%\draw [line] (Hpsi) -- 			node {} ([yshift=-3em] fbHpsi) ;
		%\draw [line] (y) -- 						node {} ([yshift=-4em] fbHy) ;
		%
		%\draw [connector] ([yshift=-0em] fbHdelta) [line]  -| node {$\delta$} (sumdelta) 			;
		%\draw [connector] ([yshift=-1em] fbHphid) [line]  -|  node {$\dot{\phi}$} (sumphid)	;
		%\draw [connector] ([yshift=-2em] fbHphi) [line]  -| 	node {$\phi$} (sumphi) 								;
		%\draw [connector] ([yshift=-3em] fbHpsi) [line]  -| 	node {$\psi$} (sumpsi) 						;
		%\draw [connector] ([yshift=-4em] fbHy) [line]  -| 	node {$y$} (sumy)										;
		%
		%\draw (sumdelta) node[below left] {$\scriptstyle-$} ;
		%\draw (sumphid) node[below left] {$\scriptstyle-$} ;
		%\draw (sumphi) node[below left] {$\scriptstyle-$} ;
		%\draw (sumpsi) node[below left] {$\scriptstyle-$} ;
		%\draw (sumy) node[below left] {$\scriptstyle-$} ;

    %\begin{pgfonlayer}{background}
        %% Compute a few helper coordinates
				%% RIder
				 %\path (Gnm.south east)+(+1em,-5.7em) node (b) {};
        %\path (Ky.north west)+(-0.5em,2em) node (a) {};
        %\path[fill=black!00,rounded corners=0.5em, draw=black!50, dashed]
						%(b) rectangle (a) node[below right, color = black!50] {Rider};
        %% Innerloop
				%\path (Gnm.south east)+(+0.5em,-3.7em) node (b) {};
				%\path (Kphi.north west)+(-0.5em,1.5em) node (a) {};
        %\path[fill=black!00,rounded corners=0.5em, draw=black!50, dashed]
						%(b) rectangle (a) node[below right, color = black!50] {Inner loop};
				%% Bike
				%\path (H.south east)+(+0.5em,-5.7em) node (b) {};
				%\path (H.north west)+(-0.5em,2em) node (a) {};
        %\path[fill=black!00,rounded corners=0.5em, draw=black!50, dashed]
						%(b) rectangle (a) node[below right, color = black!50] {Bike};					
    %\end{pgfonlayer}
    \end{tikzpicture}
\caption{Block diagram for general control models concerning roll angle control, without external excitation.}

\label{fig:hessblock}

\end{figure}